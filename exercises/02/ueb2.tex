
% Infozettel zur Vorlesung Numerik 2 WS 09/10
%
\documentclass[a4paper,10pt]{article}
\usepackage{amsmath,amssymb}
\usepackage{verbatim}
\usepackage{setspace}
\textheight 700 pt
\textwidth 450 pt
\oddsidemargin 0 pt
\evensidemargin 0 pt
\topmargin -27 pt
\headheight 0 pt
\headsep 0 pt
\parindent 0 pt
\pagestyle{empty}
\input{../mathComAbb}

\begin{document}

{\bf Otto von Guericke Universit{\"a}t Magdeburg \hfill Summer Term 2016} \\
{\bf Fakult\"at II -- Mathematik und Naturwissenschaften} \\
{\bf Institut f\"ur Mathematik} \\
Jan Heiland \hfill as of: \today \\


\bigskip
\begin{center}
\textbf{\large Differential Algebraic Equations}\\
\smallskip
\textbf{Exercise Sheet 1 -- Linear DAEs with constant coefficients}\\
\end{center}

\bigskip

% ----------------------------------------------------------------------

{\bf A Regularity and Kronecker Forms}

Check whether the matrix pairs 
\begin{equation*}
	\biggl( 
\begin{bmatrix} 1 & 1 &0 \\ 0 &-1 &1 \\ 0 &0 &0 \end{bmatrix},
\begin{bmatrix} 1  &0 &0 \\ 0 &1 &-1 \\ 0 &-1 &1 \end{bmatrix}
	\biggr)
	\andi
	\biggl( 
\begin{bmatrix} 2 &-1 &1 \\ 3 &-2 &2 \\ 0 &0 &0 \end{bmatrix},
\begin{bmatrix} 1 &0 &0 \\ 0 &1 &-1 \\ 1 &-1 &1 \end{bmatrix}
	\biggr)
\end{equation*}
are regular or singular and determine their Kronecker canonical forms by elementary row and column transforms.
\smallskip

{\bf B Index-$1$ condition}

Show that the matrix pair 
\begin{equation*}
	(E, A) = 
	\biggl( 
\begin{bmatrix} I_r &0 \\ 0 &0 \end{bmatrix}, 
	\begin{bmatrix} A_{11} & A_{12} \\ A_{21} & A_{22} \end{bmatrix}
	\biggr)
\end{equation*}
with $E$, $A\in \mathbb C^{n,n}$, and $r<n$, is of index $1$ if, and only if, $A_{22}$ is square and nonsingular.

\smallskip

{\bf C Regularity and commutativity}

Let $E$, $A\in \mathbb C^{n,n}$ satisfy $EA=AE$. Show
\begin{enumerate}
	\item that $(E, A)$ is regular if, and only if, $\kernel E \cap \kernel A = \{0\}$
	\item and that $\ind (E,A) = \ind E$.
\end{enumerate}
\smallskip

{\bf D Regularity and commutativity II}

Let $(E,A)$ be regular with $E$, $A\in \mathbb C^{n,n}$. For a $\tilde \lambda$ such that $\tilde \lambda E - A$ is invertible, show
\begin{enumerate}
	\item that $\tilde E := (\tilde \lambda E - A)^{-1}E$ and $\tilde A := (\tilde \lambda E - A)^{-1}A$ commute
	\item and that $\ind (E, A) = \ind \tilde E$.
\end{enumerate}
\smallskip

{\bf E Drazin inverse as group inverse}

If $\ind E \leq 1$, then the Drazin inverse $E^D$ is also called group inverse of $E$ and denoted by $E^\#$. Show that $E \in \mathbb C^{n,n}$ is an element of \emph{a} group $\mathbb G \subset \mathbb C^{n,n}$ with the matrix multiplication if and only if $\ind E \leq 1$, and that the inverse in such a group is just $E^\#$. (Note: The question is whether $E$ is a member of some group. I was wrong in the lecture. The Drazin inverse does not extend the group of regular matrices to matrices of index smaller or equal $1$)
\smallskip

{\bf F Drazin inverse property}

Prove that $( ( E^D)^D)^D=E^D$ for all $E\in \mathbb C^{n,n}$ 
\end{document}
