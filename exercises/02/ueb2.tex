
% Infozettel zur Vorlesung Numerik 2 WS 09/10
%
\documentclass[a4paper,10pt]{article}
\usepackage{amsmath,amssymb}
\usepackage{verbatim}
\usepackage{setspace}
\textheight 700 pt
\textwidth 450 pt
\oddsidemargin 0 pt
\evensidemargin 0 pt
\topmargin -27 pt
\headheight 0 pt
\headsep 0 pt
\parindent 0 pt
\pagestyle{empty}
%% Mathoperators
\DeclareMathOperator{\dive}{div}
\DeclareMathOperator{\grad}{grad}
\DeclareMathOperator{\spann}{span}
\DeclareMathOperator{\nulli}{null}
\DeclareMathOperator{\ind}{ind}
\DeclareMathOperator{\rank}{rank}
\DeclareMathOperator{\kernel}{kernel}
\DeclareMathOperator{\const}{const.}
\DeclareMathOperator{\trace}{tr}
\DeclareMathOperator{\image}{im}
\DeclareMathOperator{\re}{Re}



%%environments
\newcommand{\beq}[1]{\begin{equation}\label{#1}}% equation with labeling
\newcommand{\eeq}{\end{equation}} 

\providecommand{\besu}{\begin{subequations}}
\providecommand{\esu}{\end{subequations}}

%% mathsymbols
\providecommand{\abs}[1]{\lvert#1\rvert}
\providecommand{\abso}[1]{\lvert#1\rvert_1}
\providecommand{\norm}[1]{\lVert#1 \rVert}
\providecommand{\dupa}[2]{\langle #1,#2 \rangle}
\providecommand{\inva}[1]{\text{~d} #1}
\providecommand{\andi}[0]{\quad \text{and} \quad}
\providecommand{\esssup}{\mathop{\mathrm{ess\,sup}}\displaylimits}
\providecommand{\partiell}[2]{\frac{\partial #1}{\partial #2}}
\providecommand{\partielt}[1]{\frac{\partial #1}{\partial t}}
\providecommand{\timed}[1]{\frac{d #1}{dt}}
\providecommand{\into}[0]{\int_\Omega}
\providecommand{\bbmat}{\begin{bmatrix}}
\providecommand{\ebmat}{\end{bmatrix}}
\providecommand{\avr}[1]{\bigl \langle #1 \bigr \rangle}

%%Spaces
\providecommand{\Dzio}{\mathcal D _0 ^\infty (\Omega)}
\providecommand{\Lto}{\mathbf L ^2(\Omega)}
\providecommand{\lto}{ L ^2(\Omega)}
\providecommand{\Hzoo}{\mathbf H_0^1(\Omega)}
\providecommand{\hzoo}{H_0^1(\Omega)}
\providecommand{\ltzt}[1]{L^2(0,T;#1)}
\providecommand{\wzt}{\mathcal W (0,T)}

%%symbols
\providecommand{\bu}{\mathbf u}
\providecommand{\bv}{\mathbf v}
\providecommand{\bvi}{\mathbf v _ \infty}
\providecommand{\bw}{\mathbf w}
\providecommand{\vinf}{v_\infty}

% for DQMOM
\providecommand{\xia}{\xi_\alpha}
\providecommand{\xias}{\xi_\alpha^*}
\providecommand{\wia}{w_\alpha}
\providecommand{\wias}{w_\alpha^*}
\providecommand{\aia}{a_\alpha}
\providecommand{\bia}{b_\alpha}
\providecommand{\uia}{\langle u_i \rangle _\alpha}
\providecommand{\zea}{\zeta_\alpha}
\providecommand{\duba}[2]{\langle #1 \vert #2 \rangle}
\providecommand{\wxid}[2]{(w^{#1},\xi^{#2})}
\providecommand{\wxi}[1]{(w^{#1},\xi^{#1})}
%%abbrev
\providecommand{\pmo}{^{-1}}
\providecommand{\ppo}[1]{^{#1}+1}

%mschmidts
\newcommand{\BN}{{\mathbb N}}			%natural numbers
\newcommand{\BZ}{{\mathbb Z}}			%
\newcommand{\BR}{{\mathbb R}}			%real numbers
\newcommand{\BC}{{\mathbb C}}			%complex numbers
\newcommand{\sL}{{\mathscr L}} 
\newcommand{\sK}{{\mathscr K}} 
\newcommand{\sC}{{\mathscr C}} 
\newcommand{\sT}{{\mathscr T}} 
%\newcommand{\scrS}{{\mathscr S}}
\newcommand{\sP}{{\mathscr P}} 
\newcommand{\SR}{{\mathcal R}}
\newcommand{\scrS}{{\mathcal S}}
%%%% signal spaces
\newcommand{\SU}{{\mathcal U}}		
\newcommand{\SY}{{\mathcal Y}}		
\newcommand{\SZ}{{\mathcal Z}}
\newcommand{\SV}{{\mathcal V}}
\newcommand{\SW}{{\mathcal W}}
\newcommand{\BG}{\mbox{$\mathbb G$}}
\newcommand{\BF}{\mbox{$\mathbb F$}}		%input/output operator\\
\newcommand{\BP}{\mbox{$\mathbb P$}}
\newcommand{\BI}{\mbox{$\mathbb I$}}
\newcommand{\Amat}{{\bf A}}
\newcommand{\Bmat}{{\bf B}}
\newcommand{\Cmat}{{\bf C}}
\newcommand{\Dmat}{{\bf D}}
\newcommand{\Emat}{{\bf E}}	
\newcommand{\Fmat}{{\bf F}}
\newcommand{\Gmat}{{\bf G}}
\newcommand{\Hmat}{{\bf H}}
\newcommand{\Imat}{{\bf I}}
\newcommand{\Kmat}{{\bf K}}
\newcommand{\Mmat}{{\bf M}}
\newcommand{\Smat}{{\bf S}}	
\newcommand{\Umat}{{\bf U}}
\newcommand{\Vmat}{{\bf V}}		        
\newcommand{\Wmat}{{\bf W}}		        
\newcommand{\Ymat}{{\bf Y}}		        
\newcommand{\Mmass}{{\bf M}}

\newcommand{\dJ}{{\bar J}}
\newcommand{\dU}{{\bar U}}	
\newcommand{\hv}{{\bf h}}	
\newcommand{\uv}{{\bf u}}
\newcommand{\fv}{{\bf f}}

\newcommand{\zv}{{\bf z}}
\newcommand{\vv}{{\bf v}}
\newcommand{\wv}{{\bf w}}
\providecommand{\bv}{{\bf b}}
\newcommand{\yv}{{\bf y}}
\newcommand{\ev}{{\bf e}}
\providecommand{\rrn}[1]{\mathbb R ^ {#1}}


\begin{document}

{\bf Otto von Guericke Universit{\"a}t Magdeburg \hfill Summer Term 2016} \\
{\bf Fakult\"at II -- Mathematik und Naturwissenschaften} \\
{\bf Institut f\"ur Mathematik} \\
Jan Heiland \hfill as of: \today \\


\bigskip
\begin{center}
\textbf{\large Differential Algebraic Equations}\\
\smallskip
\textbf{Exercise Sheet 1 -- Linear DAEs with constant coefficients}\\
\end{center}

\bigskip

% ----------------------------------------------------------------------

{\bf A Regularity and Kronecker Forms}

Check whether the matrix pairs 
\begin{equation*}
	\biggl( 
\begin{bmatrix} 1 & 1 &0 \\ 0 &-1 &1 \\ 0 &0 &0 \end{bmatrix},
\begin{bmatrix} 1  &0 &0 \\ 0 &1 &-1 \\ 0 &-1 &1 \end{bmatrix}
	\biggr)
	\andi
	\biggl( 
\begin{bmatrix} 2 &-1 &1 \\ 3 &-2 &2 \\ 0 &0 &0 \end{bmatrix},
\begin{bmatrix} 1 &0 &0 \\ 0 &1 &-1 \\ 1 &-1 &1 \end{bmatrix}
	\biggr)
\end{equation*}
are regular or singular and determine their Kronecker canonical forms by elementary row and column transforms.
\smallskip

{\bf B Index-$1$ condition}

Show that the matrix pair 
\begin{equation*}
	(E, A) = 
	\biggl( 
\begin{bmatrix} I_r &0 \\ 0 &0 \end{bmatrix}, 
	\begin{bmatrix} A_{11} & A_{12} \\ A_{21} & A_{22} \end{bmatrix}
	\biggr)
\end{equation*}
with $E$, $A\in \mathbb C^{m,n}$, and $r<\min\{m,n\}$, is of index $1$ if, and only if, $A_{22}$ is square and nonsingular.

\smallskip

{\bf C Regularity and commutativity}

Let $E$, $A\in \mathbb C^{n,n}$ satisfy $EA=AE$. Show
\begin{enumerate}
	\item that $(E, A)$ is regular if, and only if, $\kernel E \cap \kernel A = \{0\}$
	\item and that $\ind (E,A) = \ind E$.
\end{enumerate}
\smallskip

{\bf D Regularity and commutativity II}

Let $(E,A)$ be regular with $E$, $A\in \mathbb C^{n,n}$. For a $\tilde \lambda$ such that $\tilde \lambda E - A$ is invertible, show
\begin{enumerate}
	\item that $\tilde E := (\tilde \lambda E - A)^{-1}E$ and $\tilde A := (\tilde \lambda E - A)^{-1}A$ commute
	\item and that $\ind (E, A) = \ind \tilde E$.
\end{enumerate}
\smallskip

{\bf E Drazin inverse as group inverse}

If $\ind E \leq 1$, then the Drazin inverse $E^D$ is also called group inverse of $E$ and denoted by $E^\#$. Show that $E \in \mathbb C^{n,n}$ is an element of \emph{a} group $\mathbb G \subset \mathbb C^{n,n}$ with the matrix multiplication if and only if $\ind E \leq 1$, and that the inverse in such a group is just $E^\#$. (Note: The question is whether $E$ is a member of some group. I was wrong in the lecture. The Drazin inverse does not extend the group of regular matrices to matrices of index smaller or equal $1$)
\smallskip

{\bf F Drazin inverse property}

Prove that $( ( E^D)^D)^D=E^D$ for all $E\in \mathbb C^{m,n}$ 
\end{document}
