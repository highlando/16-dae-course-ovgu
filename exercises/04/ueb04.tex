\documentclass[a4paper,10pt]{article}
\usepackage{amsmath,amssymb}
\usepackage{verbatim}
\usepackage{setspace}
\textheight 700 pt
\textwidth 450 pt
\oddsidemargin 0 pt
\evensidemargin 0 pt
\topmargin -27 pt
\headheight 0 pt
\headsep 0 pt
\parindent 0 pt
\pagestyle{empty}
\input{../mathComAbb}

\begin{document}

{\bf Otto von Guericke Universit{\"a}t Magdeburg \hfill Summer Term 2016} \\
{\bf Fakult\"at II -- Mathematik und Naturwissenschaften} \\
{\bf Institut f\"ur Mathematik} \\
Jan Heiland \hfill as of: \today \\


\bigskip
\begin{center}
\textbf{\large Differential Algebraic Equations}\\
\smallskip
\textbf{Exercise Sheet 4 -- One Step Methods}\\
\end{center}

\bigskip

% ----------------------------------------------------------------------
We consider the initial value problems 
\begin{equation}
	\dot x = f(t, x), \quad x(t_0) = x_0,
	\label{eqode}\tag{ODE}
\end{equation}
and 
\begin{equation}
	F(t, x, \dot x)=, \quad x(t_0) = x_0,
	\label{eqdae}\tag{DAE}
\end{equation}
on the time interval $[t_0, T]$. On the grid
$$t_0<t_1<\dotsm<t_N,$$
where $N\in\mathbb N$, where $t_k = t_0 + kh$, and where $h=\frac{T-t_0}{N}$, let $x_{k}$ denote a numerical approximation to $x(t_k)$, where $x$ is a solution to \eqref{eqode} or \eqref{eqdae}. We use the general form
\begin{equation*}
	\mathfrak X_{k+1} = \mathfrak F(t_{k}, \mathfrak X_{k}, h)
\end{equation*}
to refer to a given numerical scheme.

{\bf A Effect of Rounding Errors}
Consider the \emph{Explicit Euler} scheme
\begin{equation*}
	x_{k+1} = x_{k} + hf(t_{k}, x_{k})
\end{equation*}
and show that the error $e_{k} = x_{k} - y_{k}$, where $y_{k}$ are the iterates of the perturbed scheme
\begin{equation*}
	y_{k+1} = y_{k} + hf(t_{k}, y_{k}) + \varepsilon_{k},
\end{equation*}
and where $\varepsilon_{k}$ denotes the rounding error, behaves like $\norm{e_{N}} \approx \max_{k}\{\epsilon_k\}h^{-1}$.

\smallskip
{\bf B Order of Consistency}
Determine the order of consistency of the Runge--Kutta scheme
\begin{table}[h]
	\centering
	\begin{tabular}{c|cc}
		$0$ &$ 0$ & \\
		$\frac 12$ &$ \frac 12$ &$ 0$\\
		\hline
		&$0$&$1$
	\end{tabular}
\end{table}

by directly estimating $\norm{\mathfrak X(t_{k+1}) - \mathfrak F(t_{k+1}, \mathfrak X(t_{k+1}), h)}$.

\smallskip

{\bf C Two-stage Gauss method for ODEs and DAEs}
Consider the \emph{Gauss} method of order $2$:
\begin{table}[h]
	\centering
	\begin{tabular}{c|cc}
		$\frac 12 -\frac{\sqrt{3}}{6} $ &$ \frac 14$ &$\frac 14 -\frac{\sqrt{3}}{6}$ \\
		$\frac 12 +\frac{\sqrt{3}}{6}$ &$ \frac 14 + \frac{\sqrt{3}}{6}$ &$ \frac 14$\\
		\hline
		&$\frac 12$&$\frac 12$
	\end{tabular}
\end{table}
\begin{enumerate}
	\item Determine the order of convergence for ODEs using \emph{Butcher's Theorem} (lecture: Thm. 5.9).
	\item Show that $\kappa_1=2$ and $\kappa_2=2$ as defined in lecture: Thm. 5.10. What does this mean for the numerical approximation of DAEs?
\end{enumerate}
\smallskip

{\bf D Kronecker and Runge--Kutta}
Show that for the DAE with constant coefficients $E\dot x = Ax+f(t)$, the stage derivatives $\dot X_{ij}$, $j=1,\dots, s$ at time step $i$ of an $s$-stage Runge--Kutta method $(\mathcal A, \beta, \gamma)$, are defined through the linear system
\begin{equation*}
	(I_s \otimes E - h \mathcal A\otimes A) \dot X_i = Z_i,
\end{equation*}
where $\dot X_i := [\dot X_{il}]_{l=1,\dots,s}$ and $Z_i = [Ax_{i}+f(t_{i}+\gamma_lh) ]_{l=1,\dots,s}$.
\smallskip

\begin{flushright}
	Please turn the sheet.
\end{flushright}

\newpage

{\bf Coding Exercises}
\begin{enumerate}
	\item Implement the \emph{explicit Euler} scheme for constant step sizes and test it on the ODE:
		\begin{align*}
			\dot x_1 &= \phantom{-}e^tx_2, \quad x_1(0)=\sin(1),\\
			\dot x_2 &= -e^tx_1, \quad x_2(0)=\cos(1),
		\end{align*}
		through numerically computing $u_1(3)$ for stepsizes $h=3/2^k$, $k=5,6,\dots$. What do you observe? 
	\item Implement the \emph{implicit Euler} scheme for DAEs for constant step sizes as a method that takes 
		\begin{itemize}
			\item a function $F(t,x,\dot x)$,
			\item an initial value $x(t_0) = x_0$,
			\item an interval $\mathbb I = [t_0, T]$,
			\item and a number of discretization points $N$. 
		\end{itemize}
		It should return an array of the computed solution and a plot of it.
		
		Test your implementation on the following cases:
		\begin{enumerate}
			\item $F = E\dot x - Ax -f$, $x_0=[0,1]^T$, $\mathbb I = [0,10]$, and
				\begin{equation*}
					E(t) = 
					\begin{bmatrix} -t &t^2 \\ -1 &t
					\end{bmatrix}, \quad
					A(t) = 
					\begin{bmatrix} -1 &0 \\ 0 &-1
					\end{bmatrix}, \quad
					f(t) = 
					\begin{bmatrix} 0 \\ 0
					\end{bmatrix}.
				\end{equation*}
			\item $F = E\dot x - Ax -f$, $x_0=[1,1]^T$, $\mathbb I = [0,10]$, and
				\begin{equation*}
					E(t) = 
					\begin{bmatrix} 0 & 0 \\ 1 & -1
					\end{bmatrix}, \quad
					A(t) = 
					\begin{bmatrix} -1 &t \\ 0 & 0
					\end{bmatrix}, \quad
					f(t) = 
					\begin{bmatrix} \sin(t) \\ \cos(t)
					\end{bmatrix}.
				\end{equation*}
			\item $F = E\dot x - Ax -f$, $x_0=[0,1]^T$, $\mathbb I = [0,10]$, and
				\begin{equation*}
					E(t) = 
					\begin{bmatrix} 1 & 1 \\ 0 & 0
					\end{bmatrix}, \quad
					A(t) = 
					\begin{bmatrix} 0 &0 \\ 0 &-1
					\end{bmatrix}, \quad
					f(t) = 
					\begin{bmatrix} e^t+\cos(t) \\ e^t
					\end{bmatrix}.
				\end{equation*}
		\end{enumerate}
		Compare the outcomes to the actual solutions (cf. the introducing examples of Section 4 in the lecture). Are the initial values consistent? If not, determine consistent ones? What happens for inconsistent initial values?
\end{enumerate}





\end{document}
