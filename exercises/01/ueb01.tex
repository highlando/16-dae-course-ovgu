\documentclass[a4paper,10pt]{article}
\usepackage{amsmath,amssymb}
\usepackage{verbatim}
\usepackage{setspace}
\textheight 600 pt
\textwidth 450 pt
\oddsidemargin 0 pt
\evensidemargin 0 pt
% \topmargin -27 pt
\headheight 0 pt
\headsep 0 pt
\parindent 0 pt
\pagestyle{empty}
\input{../mathComAbb}

\begin{document}

{\bf Otto von Guericke Universit{\"a}t Magdeburg \hfill Summer Term 2016} \\
{\bf Fakult\"at f\"ur Mathematik} \\
{\bf Max-Planck Institut f\"ur Dynamik komplexer technischer Systeme} \\
Jan Heiland \hfill as of: \today \\


\bigskip
\begin{center}
\textbf{\large Differential Algebraic Equations}\\
\smallskip
\textbf{Exercise Sheet I -- Introductory Considerations and Basic Notions}\\
\end{center}

\bigskip

% ----------------------------------------------------------------------
{\bf A Multi-body systems}

Many multibody system can be modelled
\begin{subequations}
	\label{eq:genmbsys}
	\begin{align}
		M\dot x &= M v \\
		M\dot v &= Ax + Kv - G(x)^T\lambda + f, \\
		0&= g(x),
	\end{align}
\end{subequations}
with $x(t)\in \mathbb R^{n}$, $\lambda(t)\in \mathbb R^{m}$, $M\in \mathbb R^{n,n} $ symmetric strictly positive definite, $A\in\mathbb R^{n,n}$, $g\colon \mathbb R^{n} \to \mathbb R^{m}$, and 
$$G(x):=
\begin{bmatrix}
	\frac{\partial g_1}{\partial x_1}(x) & \hdots & \frac{\partial g_1}{\partial x_n}(x) \\
	\vdots & \ddots & \vdots \\
	\frac{\partial g_m}{\partial x_1}(x) & \hdots & \frac{\partial g_m}{\partial x_n}(x)
\end{bmatrix},
$$
is the Jacobian of $g$ at $x$, with $m<n$ and $G(x)$ having full rank for any $x$ with $g(x)=0$.
\begin{enumerate}
	\item Show that the pendulum (lecture: Exa. 1.1) can be brought into the form \eqref{eq:genmbsys}.
	\item Find, write down, and explain another multibody system that can be modelled in the form of \eqref{eq:genmbsys}. 
\end{enumerate}

\bigskip

{\bf B Separation of differential and algebraic parts}

Under certain regularity assumptions, a general DAE
\begin{equation*}
	\mathcal F(t, x, \dot x) = 0
\end{equation*}
can locally be brought into the form 
\begin{subequations}\label{eq:wellsepform}
	\begin{align}
		\dot x_1 &= \mathcal L(t, x_1, x_2), \\
		x_2 &= \mathcal R(t, x_1),
	\end{align}
\end{subequations}
by means of differentiation, elimination, and the splitting $x=[x_1,x_2]$.
\begin{enumerate}
	\item Bring the \emph{circuit} example (lecture: Exa. 1.2) into the form \eqref{eq:wellsepform} with $\mathcal L$ and $\mathcal R$ defined explicitly. How many differentiations did you need? What were the necessary regularity conditions?
	\item Bring the \emph{pendulum} example (lecture: Exa. 1.1) into the form \eqref{eq:wellsepform}. Here, $\mathcal L$ and $\mathcal R$ may be defined implicitly. How many differentiations did you need? What were the necessary regularity conditions?
	\item How can one express consistency conditions for an intial value $x_0$ by means of formulation \eqref{eq:wellsepform}.
\end{enumerate}

\bigskip

{\bf C Modelling the pendulum anew}

The pendulum can also be modelled as a pure ODE, e.g., by means of certain \emph{generalized coordinates}. Present such a model and discuss advantages of the different formulations in view of a general multibody system or \emph{Automatic Modelling} (L1).
\bigskip

{\bf D Spatially discretized linearized Navier-Stokes equations}

In simulations of flows, equations of the form 
\begin{subequations}
	\label{eq:spatlinnse}
	\begin{align}
		M \dot v &= Av - B^Tp, \quad v(0)=v_0\in \mathbb R^{n} \\
		0 &= Bv-g
	\end{align}
\end{subequations}
with $v(t) \in \mathbb R^{n}$, $p(t) \in \mathbb R^{m}$, $g(t)\in\mathbb R^{m}$, and $A$, $M \in \mathbb R^{n,n}$, and $B\in \mathbb R^{m,n}$, such that $M$ is invertible as is $BM^{-1}B^T$. Equation \eqref{eq:wellsepform} typically represents a spatially discretized and linearized Navier-Stokes equation.
\begin{enumerate}
	\item Reformulate \eqref{eq:spatlinnse} in the form of \eqref{eq:wellsepform}. How many differentiations did you need? What were the necessary regularity conditions?
	\item Write down an ODE initial value problem for $v$ and express its solution explicitly. (Hint: \emph{Variation of Constants})
	\item Prove that any $v$ that solves the initial value problem \eqref{eq:spatlinnse} also solves the ODE from {\bf D}.2. What about the converse direction?
\end{enumerate}
\smallskip


\end{document}
