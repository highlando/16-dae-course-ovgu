\documentclass[a4paper,10pt]{article}
\usepackage{amsmath,amssymb}
\usepackage{verbatim}
\usepackage{setspace}
\textheight 700 pt
\textwidth 450 pt
\oddsidemargin 0 pt
\evensidemargin 0 pt
\topmargin -27 pt
\headheight 0 pt
\headsep 0 pt
\parindent 0 pt
\pagestyle{empty}
\input{../mathComAbb}

\begin{document}

{\bf Otto von Guericke Universit{\"a}t Magdeburg \hfill Summer Term 2018} \\
{\bf Fakult\"at II -- Mathematik und Naturwissenschaften} \\
{\bf Institut f\"ur Mathematik} \\
Jan Heiland \hfill as of: \today \\


\bigskip
\begin{center}
\textbf{\large Differential Algebraic Equations}\\
\smallskip
\textbf{Exercise Sheet 3 -- Linear DAEs with Time-varying Coefficients}\\
\end{center}

\bigskip

% ----------------------------------------------------------------------

{\bf A (Local) Equivalence Transformation}

Prove that the (local) equivalence transformation defined in Definition 4.6 in the lecture defines an equivalence relation.
\smallskip

{\bf B (Global) Equivalence and Solvability}

Compute $\tilde E = PEQ$ and $\tilde A = PAQ - PE\dot Q$ for 
\begin{equation*}
	E(t) = \begin{bmatrix} 1 &0 \\ 0 &0 \end{bmatrix} ,\quad
	A(t) = \begin{bmatrix} 0 &0 \\ 1 &0 \end{bmatrix} ,\quad
	P(t) = \begin{bmatrix} t &1 \\ 1 &0 \end{bmatrix} ,\quad
	Q(t) = \begin{bmatrix} -1 &t \\ 0 &-1 \end{bmatrix} ,
\end{equation*}
and for
\begin{equation*}
	E(t) = \begin{bmatrix} 0 &1 \\ 0 &0 \end{bmatrix} ,\quad
	A(t) = \begin{bmatrix} 1 &0 \\ 0 &1 \end{bmatrix} ,\quad
	P(t) = \begin{bmatrix} 0 &-1 \\ 1 &0 \end{bmatrix} ,\quad
	Q(t) = \begin{bmatrix} 0 &-1 \\ 1 &-t \end{bmatrix} ,
\end{equation*}
compare it to the initial examples of Section 4 of the lecture, and interpret your observations.
\smallskip

{\bf C Characteristic Quantities I}

Determine the (local) characteristic quantities $(r, a, s)$ of 
\begin{equation}\label{eq:exa1}
	(E(t), A(t)) = \bigl ( \begin{bmatrix} 0 &0 \\ 1 &\eta t \end{bmatrix},
\begin{bmatrix} -1 &-\eta t  \\ 0 & -(1+\eta) \end{bmatrix}\bigr )
\end{equation}
for every $t\in\mathbb R^{}$ and for every $\eta \in \mathbb R^{}$.  
\smallskip

{\bf D Characteristic Quantities II}

Determine the (local) characteristic quantities $(r, a, s)$ of 
\begin{equation}\label{eq:exa2}
	(E(t), A(t)) = \bigl ( \begin{bmatrix} 0 &0 \\ 0 & t \end{bmatrix},
\begin{bmatrix} 1 & 0  \\ 0 & 1 \end{bmatrix}\bigr )
\end{equation}
for every $t\in\mathbb R^{}$.
\smallskip

{\bf E Drazin Inverse}

For $E\in\mathcal C(I,\mathbb R^{n,n})$, we define the Drazin inverse pointwise $E^D$ pointwise via $E^D(t)=E(t)^D$. Determine the $E^D$ for the matrix functions $E$ from \eqref{eq:exa1} and \eqref{eq:exa2}. What do you observe?
\smallskip

{\bf F Global Characteristic Quantities}

Compute the (global) characteristic quantities $(r_i, a_i, s_i)$, $i=1,\dots,\mu$, of the pair of matrix functions $(E,A)$ given in \eqref{eq:exa1}.

\end{document}
